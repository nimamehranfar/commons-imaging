\documentclass[a4paper,12pt]{report}
\usepackage{graphicx}  % for including images
\usepackage{listings}  % for displaying code
\usepackage{caption}   % for captions of images
\usepackage{amsmath}   % for math equations
\usepackage{fancyhdr}  % for header and footer customization
\usepackage[hidelinks]{hyperref}

\title{Testing and Quality Analysis of commons-imaging Project}
\author{Nima Mehranfar}
\date{\today}

\begin{document}

% Title page
\maketitle

% Abstract
\begin{abstract}
    This report presents the testing and quality analysis conducted on the commons-imaging project hosted on GitHub. The project undergoes thorough software quality analysis, CI/CD pipeline setup, code coverage analysis, performance testing, mutation testing, security assessment, and more. The results of these analyses are presented along with the actions taken to improve the quality of the commons-imaging codebase.
\end{abstract}

% Table of contents
\tableofcontents
\newpage

% Introduction
\chapter{Introduction}
\section{Background}
The commons-imaging project is a Java library developed by the Apache Software Foundation that provides an API for working with various image formats. The goal of this project is to test the commons-imaging codebase through various quality assurance practices.

\section{Objectives}
The objectives of this testing project are:
\begin{itemize}
    \item Set up CI/CD pipelines for the commons-imaging project.
    \item Perform software quality analysis using SonarCloud.
    \item Conduct code coverage analysis with Cobertura or Jacoco.
    \item Implement mutation testing with PiTest.
    \item Stress test the project components using JMH for performance analysis.
    \item Generate automated tests for poorly tested code components.
    \item Perform security analysis using OWASP FindSecBugs and OWASP DC.
    \item Create a Docker image for the project.
\end{itemize}

\newpage

% CI/CD Pipeline Setup
\chapter{CI/CD Pipeline}
The commons-imaging project has been integrated into a CI/CD pipeline that is buildable both locally and on cloud servers/github. The pipeline automatically builds, tests, and deploys the application to ensure continuous integration and delivery.

\section{CI/CD Configuration}
The configuration of the CI/CD pipeline uses GitHub Actions.
this is the file for github workflow:

\begin{lstlisting}[language=java, caption=maven.yml]
name: Java CI

on: [push, pull_request, workflow_dispatch]

permissions:
  contents: read

jobs:
  build:
    runs-on: ${{ matrix.os }}
    continue-on-error: ${{ matrix.experimental }}
    strategy:
      matrix:
        os: [windows-latest, macos-13, ubuntu-latest]
        java: [ 8, 11, 17, 21 ]
        experimental: [false]
        include:
            - java: 23
              os: ubuntu-latest
              experimental: false
            - java: 24-ea
              os: ubuntu-latest
              experimental: true

    steps:
    # https://github.com/actions/checkout/issues/226#issuecomment-854736025
    # Why is this not an issue in other Apache Commons repos?
    - name: Prepare git
      run: |-
        git config --global core.autocrlf false
        git config --global core.eol lf
    - uses: actions/checkout@11bd71901bbe5b1630ceea73d27597364c9af683 # v4.2.2
      with:
        persist-credentials: false
    - name: Set up JDK ${{ matrix.java }}
      uses: actions/setup-java@8df1039502a15bceb9433410b1a100fbe190c53b # v4.5.0
      with:
        distribution: 'temurin'
        java-version: ${{ matrix.java }}
    - name: Build with Maven
      run: mvn --errors --show-version --batch-mode --no-transfer-progress
\end{lstlisting}

\newpage

% Software Quality Analysis with SonarCloud
\chapter{Software Quality Analysis with SonarCloud}
The software quality analysis of the commons-imaging project is conducted using SonarCloud. This tool identifies potential code quality issues and provides suggestions for improvement.

\section{Issues Categorization}
The issues identified by SonarCloud are categorized as follows:
\begin{itemize}
    \item \textbf{Code Smells}: Unnecessary complexity or suboptimal code.
    \item \textbf{Bugs}: Errors in the code that could lead to failures.
    \item \textbf{Vulnerabilities}: Potential security risks in the code.
\end{itemize}

\section{Refactoring and Rationale}
A series of refactoring actions were taken to address the identified issues. For issues that were skipped, a rationale was provided. 

as the image shows there were 421 issues which 7 of them were blocker severity and 21 high severity so i fixed some issues for example:
they were lack of assertion for test cases and confusing name(duplicate) for methods and variables.
as you can see in the second image 9 issues fixed and 1 issue added because there was a test case which revealed a bug in the code (also developer of code itself commented that there was potentially a bug) so adding assertion was halting build process as shown in third image and error information below:


Assertion
[ERROR] Failures: 
[ERROR]   RationalNumberTest.testRationalNumber:120 The difference between the test value and the rational number representation exceeds the tolerance. ==> expected: <0.0> but was: <0.0020000040531158447>
.
.
.
[INFO] 
[ERROR] Tests run: 1076, Failures: 15, Errors: 0, Skipped: 7
[INFO] 

Patterns in Failures:
Small differences (<1e-9) might be acceptable due to floating-point arithmetic.
Large differences (e.g., 0.09999990463256836 or 9.223372034707292E18) are likely bugs or limitations in RationalNumber handling.
So implemention of rationalNumber.valueOf method must be checked by the developer. problem tagged as "Accepted" on SonarQube.


\section{docker}
 docker build -t nimamhr/commons-imaging .
 is not working because this project is an api and/or utility and doesn't contain main method so i created a simple webapp using react.js for frontend and java spring boot for backend.

% Docker Image Creation and Containerization
\chapter{Docker Image and Containerization}
two Docker images for the commons-imaging-webapp frontend and backend are created and pushed to DockerHub.This image is configured to be orchestrated using Docker Compose so a docker container created to run both at the same time with configured port.
these are the yaml files used to create docker images and container

\section{Dockerfile for frontend}
\begin{lstlisting}[language=java, caption=Dockerfile for commons-imaging-test-frontend]
FROM node:18-alpine AS build
WORKDIR /app
COPY package.json package-lock.json ./
RUN npm install
COPY . ./
RUN npm run build
FROM nginx:alpine
COPY --from=build /app/build /usr/share/nginx/html
EXPOSE 80
CMD ["nginx", "-g", "daemon off;"]
\end{lstlisting}

\begin{lstlisting}[language=java, caption=Dockerfile for commons-imaging-test-backend]
FROM openjdk:24-jdk
WORKDIR /app
COPY .mvn/ .mvn
COPY mvnw pom.xml ./
RUN ./mvnw dependency:go-offline -B
COPY src/ ./src
RUN ./mvnw clean package -DskipTests
EXPOSE 8080
CMD ["java", "-jar", "target/test-0.0.1-SNAPSHOT.jar"]
\end{lstlisting}

\begin{lstlisting}[language=java, caption=Example Dockerfile for container compose]
version: '3.7'

services:
  backend:
    build:
      context: ./Backend
    ports:
      - "8080:8080"
    networks:
      - app-network

  frontend:
    build:
      context: ./commons-imaging-test
    ports:
      - "80:80"
    environment:
      - REACT_APP_API_URL=http://backend:8080  # Set API URL to the backend service name
    networks:
      - app-network
    depends_on:
      - backend

networks:
  app-network:
    driver: bridge
\end{lstlisting}


also these repositories are synced with dockerhub and can be pulled and runned with commands below:

pull with:
\begin{lstlisting}[language=bash]
docker pull nimamhr/commons-imaging-webapp-frontend:latest
docker pull nimamhr/commons-imaging-webapp-backend:latest
\end{lstlisting}

and run with:
\begin{lstlisting}[language=bash]
docker run -d -p 8080:8080 nimamhr/commons-imaging-webapp-backend:latest
docker run -d -p 80:80 nimamhr/commons-imaging-webapp-frontend:latest
\end{lstlisting}

access with:
http://localhost/

\newpage

% Code Coverage Analysis
\chapter{Code Coverage Analysis}
Code coverage analysis is performed using Jacoco to determine the extent of test coverage in the commons-imaging project.

Setting up JaCoCo: adding plugin to pom.xml
\begin{lstlisting}[language=bash]
mvn clean test
mvn clean verify
\end{lstlisting}

\section{Coverage Results}
The code coverage results are as follows:
Element Missed Instructions Cov.	Missed Branches	Cov.	Missed	Cxty	Missed	Lines	Missed	Methods	Missed	Classes
Total	21,380 of 95,452	77%	2,570 of 7,199	64%	2,448	6,277	3,760	16,901	504	2,547	15	432
\begin{itemize}

 \item \textbf{}:
\end{itemize}
The final code coverage results after generatiing tests and adding some manull tests(reports are on the next steps/pages) are as follows:
Element Missed Instructions Cov.	Missed Branches	Cov.	Missed	Cxty	Missed	Lines	Missed	Methods	Missed	Classes
Total	18,077 of 95,452	81%	2,327 of 7,199	67%	2,178	6,277	3,165	16,901	345	2,547	10	432

\newpage

% Mutation Testing with PiTest
\chapter{Mutation Testing with PiTest}
Mutation testing is used to analyze the effectiveness of the test cases by introducing small modifications (mutations) to the code and ensuring that tests can detect the changes.

after adding plugins and depencencies:
mvn org.pitest:pitest-maven:mutationCoverage



- Statistics
================================================================================
>> Line Coverage (for mutated classes only): 2973/3904 (76%)
>> Generated 3564 mutations Killed 1993 (56%)
>> Mutations with no coverage 725. Test strength 70%
>> Ran 27368 tests (7.68 tests per mutation)
Enhanced functionality available at https://www.arcmutate.com/
[INFO] ------------------------------------------------------------------------
[INFO] BUILD SUCCESS
[INFO] ------------------------------------------------------------------------
[INFO] Total time:  51:10 min
[INFO] Finished at: 2025-01-03T04:29:03+01:00
[INFO] ------------------------------------------------------------------------




\section{Mutation Test Results}
The mutation test results show that:

Number of Classes	Line Coverage	Mutation Coverage	Test Strength
69	76\% 2973/3904	56\% 1993/3564	70\% 1993/2839


The number of survived mutants can be calculated as:

\begin{quote}
    Survived Mutants
=
Total Mutants -
Killed Mutants
Survived Mutants=Total Mutants-Killed Mutants
Substitute the values:

Survived Mutants=Total Mutants-Killed Mutants
Survived Mutants
=
3564 -
1993
=
1571
Survived Mutants=3564-1993=1571

\end{quote} 
\begin{itemize}
    \item \textbf{Mutants Killed}:
    \item \textbf{Surviving Mutants}:
\end{itemize}

org.apache.commons.imaging.ColorTools.java zero coverage for example: lets add some tests for this class in file ColorToolsTest
and we increased line coverage to 60\%37/62, mutation covrage to 62\%18/29	, test strength to 90\%18/20 and Surviving Mutants are 38\% 11 . not 100\% line coverage because we didn't add test for all methods.


\newpage

% Performance Testing with JMH
\chapter{Performance Testing with JMH}
Performance tests are implemented using JMH to identify performance bottlenecks in the commons-imaging project.

 Directory: C:/Users/mehra/IdeaProjects/commons-imaging/src/test/java/org/apache/commons/imaging/benchmark


Mode                 LastWriteTime         Length Name
----                 -------------         ------ ----
-a----          1/4/2025   4:04 AM           3915 ImagingBenchmark.java
-a----          1/4/2025   3:17 AM        1042592 sample.jpg
-a----          1/4/2025   3:18 AM        1006708 sample.png
-a----          1/4/2025   3:16 AM        1131930 sample.tiff

i created some benchmarks and run by these commands:
\begin{lstlisting}[language=bash]
mvn clean install
\end{lstlisting}
then
we run ImagingBenchmark class with configuration below:

\begin{lstlisting}[language=java, caption=ImagingBenchmark-configuration]
@BenchmarkMode(Mode.AverageTime)
@OutputTimeUnit(TimeUnit.MILLISECONDS)
@State(Scope.Thread)
@Fork(1)
@Warmup(iterations = 3)
@Measurement(iterations = 5)
\end{lstlisting}

\section{Stress Testing Results}
The most cumbersome components were stress-tested using JMH, and the results showed the following:

Benchmark                                    Mode  Cnt    Score   Units
ImagingBenchmark.testBmpDecoding             avgt         1.157          ms/op
ImagingBenchmark.testBmpImageInfoRetrieval   avgt         0.300          ms/op
ImagingBenchmark.testBmpImageWriting         avgt         8.348          ms/op
ImagingBenchmark.testImageInfoRetrieval      avgt         0.476          ms/op
ImagingBenchmark.testImageResizing           avgt        36.858          ms/op
ImagingBenchmark.testJpegDecoding            avgt       212.311          ms/op
ImagingBenchmark.testPngEncoding             avgt       132.896          ms/op
ImagingBenchmark.testTiffMetadataExtraction  avgt         1.217          ms/op


\begin{itemize}
    \item \textbf{Image Decoding Performance}:
    \item \textbf{writing Time}:
    ...
\end{itemize}

\newpage

% Automated Test Generation
\chapter{Automated Test Generation}
Automated tests were generated using Randoop to improve coverage on poorly tested components. These tests focus on edge cases and scenarios not previously covered.

from jacoco report we know that 3 packages (internal,formats/psd/dataparsers,formats/tiff) have low coverage as seen in image below so we want to generate test for all the classes inside them.

\begin{lstlisting}[language=bash]
cd target/classes
find org/apache/commons/imaging/internal -name "*.class" | sed 's/\//./g' | sed 's/\.class\$//'
find org/apache/commons/imaging/formats/psd/dataparsers -name "*.class" | sed 's/\//./g' | sed 's/\.class\$//'
find org/apache/commons/imaging/formats/tiff -name "*.class" | sed 's/\//./g' | sed 's/\.class\$//'
\end{lstlisting}

then pasted all the classes to Randoop\_Test\_Classes.txt file and run command below and output to randoop folder:
\begin{lstlisting}[language=bash]
java -cp "randoop-all-4.3.2.jar;target/classes" randoop.main.Main gentests $(Get-Content Randoop_Test_Classes.txt | ForEach-Object \{ "--testclass=$\_"\} ) --time-limit=200 --junit-output-dir=src/test/java/org/apache/commons/imaging/randoop/
\end{lstlisting}
with some renaming and replacing redundent classes we can now achive higher coverage as we see in the report as seen in the image below but of course this methode is not efficient enough to cover all methods with enough tests and it produces way more tests than needed and too much redundency.


\newpage

% Security Analysis with synk
\chapter{Security Analysis with synk}

steps:
\begin{lstlisting}[language=bash]
npm install -g snyk
snyk auth
snyk test
snyk test --json > snyk-report.json
\end{lstlisting}
no vulnerablity for commons imaging as seen in the image below. but there are some vulnerable pathes in the webapp backend dockerfile so i will report them instead


for example just changing the openjdk version to newer one removes so many of the vulnerablities like the image below.

\section{Security Issues Found}
The following security issues were identified: 7 low level vulnerability found and they all are Unsanitized input and seem to be false positive because this project is api and utility so there is no allowed directory and user can use any directory they want but if we want to limit user for his current directory, we can fix the all issues like the example image below Unsanitized input from a command line argument flows into listFiles, where it is used as a path. This may result in a Path Traversal vulnerability and allow an attacker to manipulate arbitrary files in class: src/test/java/org/apache/commons/imaging/examples/tiff/SurveyTiffFolder.java:collectPaths.class:
[image]

and we fix it by:

\begin{lstlisting}[language=java, caption=SurveyTiffFolder.java:collectPath.class fix]
private static final String BASE\_DIRECTORY = System.getProperty("user.dir");
private static boolean isWithinBaseDirectory(File file) throws IOException {
        // Get the canonical path of the file
        String canonicalPath = file.getCanonicalPath();
        String baseDirectoryPath = new File(BASE_DIRECTORY).getCanonicalPath();

        // Check if the canonical path starts with the base directory path
        return canonicalPath.startsWith(baseDirectoryPath);
    }
private static int collectPaths{
...
 try {
                                // Resolve the file's canonical path to get the absolute path
                                File canonicalFile = f.getCanonicalFile();

                                // Normalize the path and ensure it is within the base directory
                                if (!isWithinBaseDirectory(canonicalFile)) {
                                    continue; // Skip files outside the base directory
                                }

                                final String[] temp = Arrays.copyOf(scratch, depth + 1);
                                pathList.add(temp);
                            } catch (IOException e) {
                                // Handle any IOExceptions (e.g., if getCanonicalPath fails)
                                // e.printStackTrace();
                            }
...
}
\end{lstlisting}
all the other issues can be fixed like this if they are not false positive.

\begin{itemize}
 \item \textbf{}:
\end{itemize}

\newpage

% Conclusion
\chapter{Conclusion}
The commons-imaging project was subjected to various quality assurance practices. Through CI/CD pipeline integration, refactoring based on SonarCloud issues, performance and mutation testing, and security analysis, we have improved the project’s code quality and created a test docker for it. Future work will depend on project developers focus on addressing any remaining vulnerabilities and expanding the test coverage.

\newpage

% References
\chapter{References}
\begin{thebibliography}{99}
    \bibitem{sonarcloud} SonarCloud, \textit{SonarCloud Documentation}, \url{https://sonarcloud.io/}.
    \bibitem{docker} Docker, \textit{Docker Documentation}, \url{https://docs.docker.com/}.
    \bibitem{jacoco}
    \url{}
    \bibitem{pitest}
    \url{}
    \bibitem{jmh} JMH, \textit{JMH Documentation}, \url{https://openjdk.java.net/projects/code-tools/jmh/}.
    \bibitem{randoop}
    \url{}
    \bibitem{synk}
    \url{}
\end{thebibliography}

\end{document}
